% !TEX encoding = utf8
\documentclass[a4paper,titlepage,11pt,floatssmall]{mwrep}
\usepackage[left=2.5cm,right=2.5cm,top=2.5cm,bottom=2.5cm]{geometry}
\usepackage[T1]{fontenc}
\usepackage{polski}
\usepackage[utf8]{inputenc}

% Pakiety matematyczne
\usepackage{amsmath}
\usepackage{amsfonts}
\usepackage{amssymb}

% Pakiety graficzne i inne
\usepackage{graphicx}
\usepackage{float}
\usepackage{placeins}
\usepackage{url}
\usepackage{rotating}
\usepackage{xcolor}
\usepackage{colortbl}
\usepackage{enumitem}

% Pakiet do jednostek i liczb
\usepackage{siunitx}
\sisetup{
    detect-weight,
    range-units=single,
    output-decimal-marker={,},
    exponent-product=\cdot,
    per-mode=symbol,
    table-number-alignment = center,
    table-text-alignment = center
}

% Pakiet do listingu kodu
\usepackage{listings}
\usepackage{matlab-prettifier}

\definecolor{szary}{rgb}{0.95,0.95,0.95}

% Ustawienia globalne dla listings (dla polskich znaków i tła)
\lstset{
    literate={ą}{{\k a}}1 {Ą}{{\k A}}1 {ę}{{\k e}}1 {Ę}{{\k E}}1 {ó}{{\' o}}1 {Ó}{{\' O}}1
            {ś}{{\' s}}1 {Ś}{{\' S}}1 {ł}{{\l}}1 {Ł}{{\L}}1 {ż}{{\. z}}1 {Ż}{{\. Z}}1
            {ź}{{\' z}}1 {Ź}{{\' Z}}1 {ć}{{\' c}}1 {Ć}{{\' C}}1 {ń}{{\' n}}1 {Ń}{{\' N}}1,
    showstringspaces=false,
    backgroundcolor=\color{szary},
    breaklines=true,
    basicstyle=\footnotesize\ttfamily,
    captionpos=b
}

% Definicja stylu dla kodu MATLAB
\lstdefinestyle{custommatlab}{
    style=Matlab-editor,
    literate={ą}{{\k a}}1 {Ą}{{\k A}}1 {ę}{{\k e}}1 {Ę}{{\k E}}1 {ó}{{\' o}}1 {Ó}{{\' O}}1
            {ś}{{\' s}}1 {Ś}{{\' S}}1 {ł}{{\l}}1 {Ł}{{\L}}1 {ż}{{\. z}}1 {Ż}{{\. Z}}1
            {ź}{{\' z}}1 {Ź}{{\' Z}}1 {ć}{{\' c}}1 {Ć}{{\' C}}1 {ń}{{\' n}}1 {Ń}{{\' N}}1,
    backgroundcolor=\color{szary},
    breaklines=true,
    basicstyle=\footnotesize\ttfamily,
    captionpos=b
}

% Polskie nazwy dla rysunków i tabel
\def\figurename{Rys.}
\def\tablename{Tab.}
\def\lstlistingname{Listing}

% Poprawki do składu obiektów pływających
\setcounter{topnumber}{2}
\setcounter{bottomnumber}{2}
\setcounter{totalnumber}{4}
\renewcommand{\textfraction}{0.07}
\renewcommand{\topfraction}{0.9}
\renewcommand{\bottomfraction}{0.8}
\renewcommand{\floatpagefraction}{0.7}


% --- Dane do strony tytułowej ---
\title{\bf Sprawozdanie z laboratorium \\ Sterowanie Procesami \\ Temat: Regulatory predykcyjne}
\author{Jakub Szubzda}
\date{\today}

% --- Definicja strony tytułowej ---
\makeatletter
\renewcommand{\maketitle}{\begin{titlepage}
\begin{center}{\LARGE {\bf
Wydział Elektroniki i~Technik Informacyjnych}}\\
\vspace{0.4cm}
{\LARGE {\bf Politechnika Warszawska}}\\
\vspace{0.3cm}
\end{center}
\vspace{5cm}
\begin{center}
{\bf \LARGE Przedmiot: Sterowanie procesami \vskip 0.1cm}
\end{center}
\vspace{1cm}
\begin{center}
{\bf \LARGE \@title \vskip 0.1cm}
\end{center}
\vspace{2cm}
\begin{center}
{\bf \Large Autor: \@author \par}
\end{center}
\vspace*{\stretch{6}}
\begin{center}
\bf{\large{Warszawa, \@date\vskip 0.1cm}}
\end{center}
\end{titlepage}
}
\makeatother

\begin{document}
\frenchspacing
\pagestyle{uheadings}

\maketitle

\tableofcontents

\chapter{Wstęp}

Celem projektu laboratoryjnego było zapoznanie się z~regulatorami predykcyjnymi DMC (Dynamic Matrix Control) i~GPC (Generalized Predictive Control) oraz porównanie ich działania z~klasycznym regulatorem PID. Wykonano analizę obiektu inercyjnego drugiego rzędu z~opóźnieniem, zaprojektowano dla niego regulatory predykcyjne oraz przeprowadzono symulacje w~środowisku MATLAB/Simulink.

W ramach projektu zrealizowano następujące zadania:
\begin{enumerate}
    \item Analiza transmitancji obiektu oraz jej reprezentacja w~postaci dyskretnej
    \item Wyprowadzenie równania różnicowego na podstawie transmitancji dyskretnej
    \item Strojenie regulatora PID metodą Zieglera-Nicholsa
    \item Synteza i~badanie regulatora DMC (wpływ horyzontu predykcji, horyzontu sterowania oraz współczynnika kary za sterowanie)
    \item Porównanie regulatorów DMC i~PID
    \item Porównanie regulatorów DMC i~GPC
    \item Badanie obszarów stabilności regulatorów PID, DMC i~GPC
\end{enumerate}

Projekt pozwolił na praktyczne poznanie zalet i~ograniczeń regulatorów predykcyjnych oraz nabycie umiejętności doboru ich parametrów.

\chapter{Zadania projektowe}

\section{Analiza transmitancji obiektu}

Obiektem badań był układ inercyjny drugiego rzędu z~opóźnieniem, opisany transmitancją ciągłą:
\begin{equation}
    G(s) = \frac{K_o e^{-T_o s}}{(T_1 s + 1)(T_2 s + 1)}
\end{equation}

gdzie:
\begin{itemize}
    \item $K_o = 4,7$ - współczynnik wzmocnienia statycznego
    \item $T_o = 5$ s - opóźnienie transportowe
    \item $T_1 = 1,92$ s - pierwsza stała czasowa
    \item $T_2 = 4,96$ s - druga stała czasowa
\end{itemize}

Po podstawieniu wartości parametrów uzyskano transmitancję ciągłą w~postaci:
\begin{equation}
    G(s) = e^{-5s} \cdot \frac{4,7}{9,523s^2 + 6,88s + 1}
\end{equation}

Transmitancja dyskretna została wyznaczona za pomocą funkcji \texttt{c2d(G,Tp,'zoh')} w~środowisku MATLAB, z~okresem próbkowania $T_p = 0,5$ s. Uzyskano następującą postać:
\begin{equation}
    G(z) = z^{-10} \cdot \frac{0,05477z + 0,04856}{z^2 - 1,675z + 0,6968}
\end{equation}

Współczynniki wzmocnienia statycznego dla obu transmitancji są równe $K_s = K_z = 4,7$, co potwierdza poprawność przekształcenia z postaci ciągłej na dyskretną.

\begin{figure}[H]
    \centering
    \includegraphics[width=0.8\textwidth]{kod/wykresy/zad1_porownanie_odpowiedzi_skokowych.jpg}
    \caption{Porównanie odpowiedzi skokowych obiektu ciągłego $G(s)$ i~dyskretnego $G(z)$ dla okresu próbkowania $T_p = 0,5$ s}
    \label{fig:odp_skokowe}
\end{figure}

Na rysunku \ref{fig:odp_skokowe} widać, że odpowiedzi skokowe modelu ciągłego i~dyskretnego są zbliżone, co potwierdza poprawność dyskretyzacji.

\section{Wyprowadzenie równania różnicowego}

Na podstawie transmitancji dyskretnej $G(z)$ wyprowadzono równanie różnicowe opisujące obiekt. Transmitancja dyskretna ma postać:
\begin{equation}
    G(z) = z^{-10} \cdot \frac{0,05477z + 0,04856}{z^2 - 1,675z + 0,6968}
\end{equation}

Równanie różnicowe wyprowadzone z~tej transmitancji:
\begin{equation}
    y(k) = 1,6748 \cdot y(k-1) - 0,69682 \cdot y(k-2) + 0,054771 \cdot u(k-11) + 0,048559 \cdot u(k-12)
\end{equation}

Ta postać pozwala na rekurencyjne obliczanie wyjścia procesu na podstawie poprzednich wartości wyjścia i~sterowania. Opóźnienie transportowe jest reprezentowane przez przesunięcie sygnału wejściowego o~10 okresów próbkowania ($u(k-10)$ i~później).

\section{Strojenie regulatora PID metodą Zieglera-Nicholsa}

Regulator PID został dostrojony metodą Zieglera-Nicholsa z~wykorzystaniem parametrów krytycznych. Na podstawie analizy charakterystyki częstotliwościowej obiektu wyznaczono następujące parametry krytyczne:
\begin{itemize}
    \item Wzmocnienie krytyczne $K_k = 0,46514$
    \item Okres oscylacji krytycznych $T_k = 10,6074$ s
    \item Pulsacja krytyczna $\omega_k = 0,59234$ rad/s
\end{itemize}

Zgodnie z~zasadami Zieglera-Nicholsa, obliczono parametry ciągłego regulatora PID:
\begin{itemize}
    \item $K_r = 0,19536$ (współczynnik wzmocnienia)
    \item $T_i = 7,5767$ s (stała czasowa całkowania)
    \item $T_d = 0,89102$ s (stała czasowa różniczkowania)
\end{itemize}

Co odpowiada następującym współczynnikom $K_p$, $K_i$ i~$K_d$:
\begin{itemize}
    \item $K_p = 0,19536$ (wzmocnienie członu proporcjonalnego)
    \item $K_i = 0,025784$ (wzmocnienie członu całkującego)
    \item $K_d = 0,17407$ (wzmocnienie członu różniczkującego)
\end{itemize}

Na podstawie tych wartości wyznaczono parametry dyskretnego regulatora PID w~postaci równania różnicowego:
\begin{equation}
    u(k) = u(k-1) + r_0 e(k) + r_1 e(k-1) + r_2 e(k-2)
\end{equation}

gdzie:
\begin{itemize}
    \item $r_0 = 0,55639$
    \item $r_1 = -0,89163$
    \item $r_2 = 0,34814$
\end{itemize}

System wyświetlił ostrzeżenie, że układ zamknięty z~otrzymanymi parametrami może być niestabilny, co wskazuje na trudności w~regulacji tego obiektu przy użyciu standardowego regulatora PID. Jest to typowe dla obiektów z~dużym opóźnieniem transportowym.

\section{Synteza i~badanie regulatora DMC}

\subsection{Wyznaczenie horyzontu dynamiki}

Pierwszym krokiem w~syntezie regulatora DMC było wyznaczenie horyzontu dynamiki $D$, czyli liczby współczynników odpowiedzi skokowej koniecznych do poprawnego zamodelowania dynamiki obiektu. Na podstawie odpowiedzi skokowej obiektu (rysunek \ref{fig:step}) przyjęto $D = 60$, co odpowiada czasowi $30$ sekund (przy $T_p = 0,5$ s), po którym odpowiedź skokowa praktycznie osiąga stan ustalony.

\begin{figure}[H]
    \centering
    \includegraphics[width=0.7\textwidth]{kod/wykresy/step.jpg}
    \caption{Odpowiedź skokowa obiektu dyskretnego wykorzystana do wyznaczenia horyzontu dynamiki $D$}
    \label{fig:step}
\end{figure}

\subsection{Badanie wpływu horyzontu predykcji $N$}

W regulatorze DMC horyzont predykcji $N$ określa, na ile kroków w~przyszłość prognozowane jest zachowanie obiektu. Aby zbadać wpływ tego parametru na jakość regulacji, przeprowadzono symulację dla różnych wartości $N$ przy ustalonych pozostałych parametrach. Na rysunku \ref{fig:N_comparison} przedstawiono porównanie odpowiedzi układu dla wartości $N \in \{20, 30, 40, 50, 60\}$, przy założeniu $N_u = N$ i~$\lambda = 1$.

\begin{figure}[H]
    \centering
    \includegraphics[width=0.8\textwidth]{kod/wykresy/horyzont_predykcji_porownanie.jpg}
    \caption{Wpływ horyzontu predykcji $N$ na jakość regulacji (przy $N_u = N$)}
    \label{fig:N_comparison}
\end{figure}

Z~analizy wynika, że dla $N < 30$ układ wykazuje słabsze tłumienie oscylacji. Przy większych wartościach odpowiedzi stają się bardziej stabilne, ale przy $N > 40$ dalsze zwiększanie horyzontu predykcji nie przynosi już znaczącej poprawy. Jako optymalną wybrano wartość $N = 20$.

\subsection{Badanie wpływu horyzontu sterowania $N_u$}

Horyzont sterowania $N_u$ określa, na ile kroków w~przyszłość obliczane są przyrosty sterowania. Aby zbadać wpływ tego parametru, przeprowadzono symulacje dla stałego horyzontu predykcji $N = 20$ i~różnych wartości $N_u \in \{1, 5, 10, 20, 40, 60\}$.

\begin{figure}[H]
    \centering
    \includegraphics[width=0.8\textwidth]{kod/wykresy/horyzont_sterowania_porownanie.jpg}
    \caption{Wpływ horyzontu sterowania $N_u$ na jakość regulacji (przy $N = 20$)}
    \label{fig:Nu_comparison}
\end{figure}

Z~rysunku \ref{fig:Nu_comparison} wynika, że dla małych wartości $N_u$ (szczególnie $N_u = 1$) odpowiedź układu jest wolniejsza, ale za to łagodniejsza. Wraz ze wzrostem $N_u$ odpowiedź staje się szybsza, ale pojawia się przesterowanie i~większe oscylacje sterowania. Jako kompromis między szybkością a~jakością wybrano $N_u = 5$.

\subsection{Badanie wpływu współczynnika kary za sterowanie $\lambda$}

Współczynnik $\lambda$ odpowiada za wagę członu kary za przyrosty sterowania w~funkcji celu regulatora DMC. Im większa wartość $\lambda$, tym większa kara za duże zmiany sterowania, co prowadzi do łagodniejszego działania regulatora.

\begin{figure}[H]
    \centering
    \includegraphics[width=0.8\textwidth]{kod/wykresy/lambda_porownanie.jpg}
    \caption{Wpływ współczynnika kary $\lambda$ na jakość regulacji (przy $N = 20$, $N_u = 5$)}
    \label{fig:lambda_comparison}
\end{figure}

Z~rysunku \ref{fig:lambda_comparison} wynika, że dla małych wartości $\lambda$ (np. $0,1$) układ reaguje szybko, ale sterowanie może być agresywne. Dla większych wartości $\lambda$ (np. $5$ czy $10$) odpowiedź jest wolniejsza i~łagodniejsza. Jako optymalną wartość wybrano $\lambda = 0,1$, która zapewnia szybką odpowiedź przy akceptowalnym profilu sterowania.

Po przeprowadzeniu powyższych badań ustalono optymalne parametry regulatora DMC:
\begin{itemize}
    \item Horyzont predykcji $N = 20$
    \item Horyzont sterowania $N_u = 5$
    \item Współczynnik kary $\lambda = 0,1$
\end{itemize}

\section{Porównanie regulatorów DMC i~PID}

Po określeniu optymalnych parametrów regulatora DMC przeprowadzono porównanie jego działania z~regulatorem PID dostrojonym metodą Zieglera-Nicholsa. Rysunek \ref{fig:dmc_pid} przedstawia porównanie odpowiedzi układu oraz sygnałów sterujących dla obu regulatorów.

\begin{figure}[H]
    \centering
    \includegraphics[width=0.8\textwidth]{kod/wykresy/optimal_solution.jpg}
    \caption{Porównanie odpowiedzi układu z~regulatorem DMC i~PID}
    \label{fig:dmc_pid}
\end{figure}

Regulator DMC wykazuje lepszą jakość regulacji niż PID - charakteryzuje się mniejszym przeregulowaniem, szybszym czasem regulacji i~lepszym tłumieniem oscylacji. Sygnał sterujący generowany przez DMC ma łagodniejszy przebieg niż w~przypadku PID, co jest korzystne z~punktu widzenia elementów wykonawczych.

\section{Porównanie regulatorów DMC i~GPC}

\subsection{Reakcja na zmianę wartości zadanej}

Porównano działanie regulatorów DMC i~GPC przy jednakowych parametrach ($N = 20$, $N_u = 5$, $\lambda = 0,1$). Na rysunku \ref{fig:dmc_gpc_setpoint} przedstawiono odpowiedź układu na zmianę wartości zadanej.

\begin{figure}[H]
    \centering
    \includegraphics[width=0.8\textwidth]{kod/wykresy/DMC_vs_GPC_setpoint.jpg}
    \caption{Porównanie odpowiedzi układu z~regulatorami DMC i~GPC na zmianę wartości zadanej}
    \label{fig:dmc_gpc_setpoint}
\end{figure}

Zarówno regulator DMC, jak i~GPC zapewniają podobną jakość regulacji, jednak GPC charakteryzuje się nieco szybszą odpowiedzią i~mniejszym przeregulowaniem. Wynika to z~różnego sposobu modelowania obiektu w~tych algorytmach - DMC wykorzystuje model odpowiedzi skokowej, a~GPC - model w~postaci równania różnicowego.

\subsection{Reakcja na zakłócenie}

Przeprowadzono również test odporności na zakłócenie działające na wyjściu obiektu. Na rysunku \ref{fig:dmc_gpc_disturbance} przedstawiono odpowiedź układu na zakłócenie o~wartości $0,3$.

\begin{figure}[H]
    \centering
    \includegraphics[width=0.8\textwidth]{kod/wykresy/DMC_vs_GPC_disturbance.jpg}
    \caption{Porównanie odpowiedzi układu z~regulatorami DMC i~GPC na zakłócenie}
    \label{fig:dmc_gpc_disturbance}
\end{figure}

W przypadku zakłócenia regulator GPC wykazuje szybszą reakcję i~lepsze tłumienie jego wpływu. Regulator DMC również skutecznie kompensuje zakłócenie, ale potrzebuje nieco więcej czasu. Wynika to z~różnych modeli procesu i~algorytmu predykcji zakłóceń.

\section{Badanie obszarów stabilności}

W ostatnim etapie projektu przeprowadzono badanie obszarów stabilności regulatorów PID, DMC i~GPC w~funkcji zmiany parametrów obiektu: wzmocnienia $K_o$ i~opóźnienia transportowego $T_o$. Zbadano wartości krytyczne wzmocnienia $K_o$ dla różnych wartości opóźnienia $T_o$, przy których układ regulacji traci stabilność.

Wyniki symulacji:
\begin{itemize}
    \item To=5,00, Ko\_kryt: GPC=18,40, DMC=13,60, PID=10,40
    \item To=5,50, Ko\_kryt: GPC=18,50, DMC=13,80, PID=9,60
    \item To=6,00, Ko\_kryt: GPC=19,40, DMC=13,90, PID=9,10
    \item To=6,50, Ko\_kryt: GPC=22,00, DMC=14,00, PID=8,70
    \item To=7,00, Ko\_kryt: GPC=31,60, DMC=14,00, PID=8,20
    \item To=7,50, Ko\_kryt: GPC=30,40, DMC=14,30, PID=7,70
    \item To=8,00, Ko\_kryt: GPC=43,70, DMC=15,70, PID=7,30
    \item To=8,50, Ko\_kryt: GPC=2,60, DMC=12,50, PID=7,00
    \item To=9,00, Ko\_kryt: GPC=4,20, DMC=21,50, PID=7,00
    \item To=9,50, Ko\_kryt: GPC=0,00, DMC=0,00, PID=6,90
    \item To=10,00, Ko\_kryt: GPC=0,00, DMC=0,00, PID=6,70
\end{itemize}

Na rysunku \ref{fig:stability} przedstawiono otrzymane krzywe graniczne stabilności.

\begin{figure}[H]
    \centering
    \includegraphics[width=0.8\textwidth]{kod/wykresy/GPC_DMC_PID_stabilnosc_Ko_vs_To.jpg}
    \caption{Krzywe graniczne stabilności dla regulatorów GPC, DMC i~PID w~funkcji opóźnienia $T_o$ i~wzmocnienia $K_o$}
    \label{fig:stability}
\end{figure}

Na osiach wykresu przedstawiono względne zmiany parametrów $T_o/T_{o,nom}$ oraz $K_o/K_{o,nom}$. Obszar stabilności leży poniżej krzywych dla każdego z~regulatorów - dla danego opóźnienia $T_o/T_{o,nom}$, system jest stabilny jeśli $K_o/K_{o,nom} < K_{o,kryt}/K_{o,nom}$.

Widoczne jest, że regulatory predykcyjne (DMC i~GPC) mają znacznie szerszy obszar stabilności niż regulator PID, szczególnie dla większych opóźnień. Regulator GPC wykazuje największą odporność na wzrost opóźnienia transportowego, co jest jego istotną zaletą w~zastosowaniach przemysłowych.

Warto odnotować, że dla bardzo dużych opóźnień (To > 9,0) wszystkie regulatory tracą stabilność nawet przy niewielkim wzroście wzmocnienia obiektu. Jest to zgodne z~intuicją, ponieważ duże opóźnienie transportowe jest jednym z~najtrudniejszych wyzwań w~regulacji.

\chapter{Wnioski}

Na podstawie przeprowadzonych badań można sformułować następujące wnioski:

\begin{enumerate}
    \item Regulatory predykcyjne (DMC i~GPC) zapewniają lepszą jakość regulacji niż klasyczny regulator PID, szczególnie dla obiektów z~opóźnieniem transportowym. Wykazują mniejsze przeregulowanie, krótszy czas regulacji i~lepsze tłumienie oscylacji.

    \item Parametry regulatora DMC ($N$, $N_u$, $\lambda$) mają istotny wpływ na jakość regulacji. Horyzont predykcji $N$ powinien być wystarczająco duży, aby uwzględnić pełną odpowiedź obiektu. Horyzont sterowania $N_u$ wpływa na agresywność sterowania - większe wartości prowadzą do szybszej, ale bardziej oscylacyjnej odpowiedzi. Współczynnik $\lambda$ pozwala na kompromis między szybkością regulacji a~łagodnością sterowania.

    \item Regulator GPC wykazuje lepsze właściwości niż DMC w~zakresie kompensacji zakłóceń oraz odporności na zmiany parametrów obiektu, szczególnie opóźnienia transportowego. Wynika to z~odmiennego sposobu modelowania obiektu i~predykcji.

    \item Obszar stabilności regulatorów predykcyjnych jest znacznie szerszy niż regulatora PID, co czyni je bardziej odpornymi na zmiany parametrów obiektu. Jest to szczególnie istotne w~przypadku procesów o~zmiennych parametrach lub niepewności modelowania.
\end{enumerate}

Podsumowując, regulatory predykcyjne stanowią efektywne narzędzie do sterowania procesami, szczególnie w~przypadku obiektów trudnych (z opóźnieniem, niestabilnych, wielowymiarowych). Ich implementacja wymaga jednak dokładniejszego modelowania procesu oraz większej mocy obliczeniowej niż w~przypadku klasycznych regulatorów PID.

\chapter{Kod źródłowy}

Poniżej przedstawiono listę skryptów MATLAB wykorzystanych w projekcie. Szczegółowe treści tych skryptów znajdują się w katalogu \texttt{/home/jszubzda/STP\_2/kod/}.

\section{Główny skrypt projektu}

\lstinputlisting[style=custommatlab, caption={Główny skrypt projektu - stp2.m}, label={lst:main}]{kod/stp2.m}

\section{Funkcje do analizy transmitancji i wyprowadzenia równania różnicowego}

\lstinputlisting[style=custommatlab, caption={zadanie1\_analiza\_transmitancji.m}, label={lst:zadanie1}]{kod/zadanie1_analiza_transmitancji.m}

\lstinputlisting[style=custommatlab, caption={zadanie2\_rownanie\_roznicowe.m}, label={lst:zadanie2}]{kod/zadanie2_rownanie_roznicowe.m}

\section{Funkcje do strojenia regulatora PID}

\lstinputlisting[style=custommatlab, caption={zadanie3\_strojenie\_pid\_ziegler\_nichols.m}, label={lst:zadanie3}]{kod/zadanie3_strojenie_pid_ziegler_nichols.m}

\section{Funkcje do projektowania i badania regulatora DMC}

\lstinputlisting[style=custommatlab, caption={zadanie5\_optymalizacja\_dmc.m}, label={lst:zadanie5}]{kod/zadanie5_optymalizacja_dmc.m}

\lstinputlisting[style=custommatlab, caption={zadanie6\_porownanie\_optymalne\_dmc\_pid.m}, label={lst:zadanie6}]{kod/zadanie6_porownanie_optymalne_dmc_pid.m}

\section{Funkcje do porównania regulatorów DMC i GPC}

\lstinputlisting[style=custommatlab, caption={zadanie8\_porownanie\_dmc\_gpc.m}, label={lst:zadanie8}]{kod/zadanie8_porownanie_dmc_gpc.m}

\section{Funkcje do badania stabilności}

\lstinputlisting[style=custommatlab, caption={zadanie9\_badanie\_stabilnosci.m}, label={lst:zadanie9}]{kod/zadanie9_badanie_stabilnosci.m}

\lstinputlisting[style=custommatlab, caption={czy\_oscyluje.m}, label={lst:czy_oscyluje}]{kod/czy_oscyluje.m}

\section{Funkcje pomocnicze do symulacji}

\lstinputlisting[style=custommatlab, caption={symulacja\_dmc\_pid.m}, label={lst:symulacja_dmc_pid}]{kod/symulacja_dmc_pid.m}

\lstinputlisting[style=custommatlab, caption={symulacja\_gpc.m}, label={lst:symulacja_gpc}]{kod/symulacja_gpc.m}

\end{document}
